Atualmente, a principal forma de contratação relacionada a mão de obra especializada é realizada por meio de indicações ou anúncios, onde contratantes avaliam os candidatos com base em descrições e portfólios apresentados. No entanto, a seleção final geralmente é confirmada após conversas diretas ou pequenas tarefas iniciais para validar a adequação do profissional ao trabalho requerido, logo esse processo acaba levando um tempo maior para ser concluído, e muitas vezes existe a necessidade de uma solução mais ágil para solucionar tais demandas de serviço.\\

Este projeto visa como objetivo principal desenvolver uma aplicação para contratação de freelancers que seja eficiente e intuitiva, agregando aos profissionais autônomos uma ferramenta essêncial para divulgação de seu trabalho. Através desta aplicação, buscamos facilitar o processo de contratação de serviços especializados, permitindo que contratantes encontrem soluções adequadas para suas necessidades e que freelancers tenham acesso a uma ampla gama de oportunidades de trabalho. Para isso temos alguns outros objetivos a serem alcançados:


\subsection{Conexão e Dinâmica}
\subsubsection{Desenvolver uma aplicação que conecte contratantes e prestadores de serviços, facilitando a intermediação entre freelancers e empresas ou indivíduos que necessitam de serviços especializados. A plataforma será projetada para otimizar a busca e seleção de profissionais, oferecendo uma interface intuitiva e ferramentas de gestão que permitam aos usuários administrar suas contratações e projetos de maneira prática e organizada.}

\subsection{Pagamentos} 
\subsubsection{Criação de um ambiente seguro para a realização de transações financeiras diretamente pelo aplicativo, garantindo a segurança e eficiência entre contratantes e mão de obra qualificada. O sistema de pagamento integrado eliminará a necessidade de intermediários externos, oferecendo maior controle sobre o fluxo financeiro e assegurando a proteção dos dados dos usuários, com foco na transparência e na confiabilidade das operações.}

\subsection{Segurança e Confiabilidade}
\subsubsection{Proporcionar um ambiente seguro para a contratação de serviços, implementando mecanismos avançados de segurança como verificações de identidade, avaliações de usuários e medidas contra fraudes e abusos. Foco no desenvolvimento de uma plataforma confiável, onde tanto contratantes quanto freelancers possam realizar transações e acordos de trabalho com segurança e tranquilidade.}

\subsection{Ecossistema e Geolocalização} 
\subsubsection{Projetar a aplicação para fomentar um ecossistema de trabalho colaborativo e inclusivo, incentivando a diversidade entre profissionais de diferentes áreas e regiões geográficas. A plataforma terá recursos que facilitam a comunicação e a colaboração entre as partes, permitindo dessa maneira uma seleção mais eficaz dos profissionais localizados próximo ao local do contratante, poupando tempo para ambas as partes envolvidas e promovendo uma alta rotatividade de serviços locais.}
