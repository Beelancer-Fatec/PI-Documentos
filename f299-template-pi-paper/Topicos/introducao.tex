
A Organização das Nações Unidas (ONU) foi fundada com o objetivo de reunir todas as nações a fim de discutir problemas e encontrar soluções palpáveis, a instituição conta com 193 estados-membros, que representam no órgão deliberativo a Assembleia Geral. Ao fim do ano de 2015, a ONU desenvolveu a Agenda 2030, um plano de ação global contendo 17 metas destinadas a sustentabilidade, denominada Objetivos de Desenvolvimento Sustentável (ODS). Os objetivos e metas são integrados e possuem três dimensões principais, sendo estas: social, ambiental e econômica. Este trabalho busca auxiliar o alcance das metas apresentadas pela Agenda 2030 abordando os seguintes tópicos: Promover o crescimento econômico inclusivo e sustentável, o emprego pleno e produtivo e o trabalho digno para todos, associado à oitava meta, que promove Trabalho decente e crescimento econômico.\\

O mercado de trabalho vem evoluindo de forma exponencial nas últimas décadas, sendo marcado pelo surgimento e crescimento da gig economy termo que refere-se ao modelo caracterizado pela prestação de serviços temporários por autônomos, essa dinâmica está majorietariamente relacionada a contratação de freelancers como mão de obra, resultando na busca crescente por plataformas que otimizem a intermediação entre contratantes e profissionais qualificados. Diante desse cenário, surge a necessidade de desenvolver ferramentas que facilitem a conexão entre contratantes e prestadores de serviços especializados, oferecendo uma solução eficaz para o dinamismo atual do mercado de trabalho, tal como a evolução da economia digital tem sido determinante em sua transformação e fomentação de novas formas de interação entre profissionais e empregadores, com a popularização de plataformas digitais especializadas, como Uber e Fiverr, demonstra-se o impacto da tecnologia na facilitação de contratações temporárias, oferecendo não só praticidade, mas também escalabilidade às relações de trabalho, no entanto, grande parte dessas plataformas concentram-se em nichos específicos ou apresentam limitações quanto à diversidade dos serviços ofertados e à qualificação dos profissionais disponíveis, evidenciando uma lacuna no mercado.\\

A criação do aplicativo Bee-lancer, tem como foco principal a diversidade de serviços qualificados em múltiplas áreas da economia nacional ou internacional, apresenta-se como uma solução estratégica diante dessa necessidade, contribuindo para um ambiente mais organizado e acessível para contratantes que necessitam de mão de obra e profissionais que buscam oportunidades adequadas às suas habilidades. Estudos sobre a gig economy, como os de \cite{Kassie2018}, apontam a importância de plataformas que sejam capazes de garantir não apenas a intermediação, mas também um ecossistema de confiança, com avaliações de desempenho e garantias contratuais. Este projeto propõe uma plataforma ágil e inclusiva, onde os contratantes podem facilmente identificar e contatar profissionais que oferecem seus serviços com segurança.\\

O problema central a ser abordado, é a dificuldade de se encontrar profissionais de forma ágil e também a enfrentada por trabalhadores independentes em se destacarem no mercado atual competitivo e fragmentado, o estudo tem como foco desenvolver e analisar o impacto de um aplicativo que atue como facilitador na conexão entre contratantes e prestadores de serviços, promovendo uma interação eficiente, transparente e baseada em critérios de qualidade e confiabilidade, logo, este trabalho traz contribuições significativas para a área de estudo das plataformas digitais. O Bee-lancer propõe um modelo que vai além da simples intermediação, integrando mecanismos de verificação de competências e qualidade de serviço, o que aumenta a confiança entre as partes envolvidas nessas dinâmicas, além disso, oferece uma solução que se adapta às exigências de um mercado de trabalho em constante evolução, promovendo maior inclusão de profissionais especializados e maior eficiência na contratação de serviços, aspectos que têm sido discutidos amplamente na literatura recente sobre economia digital e mercados de trabalho alternativos.\\