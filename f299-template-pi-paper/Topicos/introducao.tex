No final de 2015, a Organização das Nações Unidas (ONU) desenvolveu a Agenda 2030, um plano de ação global contendo 17 metas destinadas a promoção da sustentabilidade, denominada Objetivos de Desenvolvimento Sustentável (ODS). Este trabalho está fortemente ligado ao alcance das metas apresentadas pela agenda abordando os seguintes tópicos: Promover o crescimento econômico inclusivo e sustentável, o emprego pleno e produtivo e o trabalho digno para todos, associado à ODS 8, que promove Trabalho decente e crescimento econômico. 

Nesse contexto, destaca-se a ascensão da gig economy, um termo que se refere ao modelo de trabalho caracterizado pela contração de serviços autônomos, com contratos flexíveis e de curta duração. Este modelo abrange uma variedade de profissões, como motoristas de aplicativos, entregadores, profissionais autônomos de serviços gerais e até trabalhadores especializados, como designers e programadores. A consolidação definitiva desse modelo teve início por volta de 1995, impulsionada pela evolução da internet e pelo surgimento de plataformas de contratação de serviços como CraigList, Elance, Odesk \cite{ROCKCONTENT2023} e posteriormente o desenvolvimento dos aplicativos Uber e Ifood.  

No Brasil, essa realidade é bastante evidente. No segundo trimestre de 2022, a taxa de informalidade atingiu 40\% \cite{IBGE2022}, esse percentual engloba profissionais vinculados à gig economy. Entre estes, destacam-se prestadores de serviços gerais, como pedreiros, pintores, encanadores, eletricistas, marceneiros e jardineiros. Esses trabalhadores enfrentam desafios significativos, como dificuldades para serem contratados, baixa visibilidade, carência de valorização profissional e ausência de garantias legais, o que contribui para um mercado mais vulnerável e carente de melhorias estruturais.  

Nesse sentido é importante destacar que a modalidade da economia gig tem sido objeto de debates, principalmente no que diz respeito aos interesses dos profissionais e das empresas intermediadoras. Conforme discutido por Rocha \cite{RochaFerreira2022}, o fenômeno de contratos flexíveis mediados por plataformas evidencia a necessidade de consolidação de proteções aos trabalhadores, apontando problemáticas como a falta de direitos trablhistas, remunerações desproporcionais e a utilização de algoritmos considerados não transparentes. 

Além do mais, sob a perspectiva dos contratantes, é possível observar ausências de garantias relacionadas à qualidade do serviço prestado, e, no caso específico de mulheres, destaca-se a questão de insegurança ao receberem profissionais desconhecidos em suas residências \cite{ESTADAO2020} 

Diante desse cenário evidencia-se a necessidade de mecanismo intermediadores que facilitem a contratação entre trabalhadores inseridos na gig economy e os contratantes, de forma transparente, segura e que valorize o trabalhador. Tais ferramentas podem contribuir para garantir maior segurança, formalidade e confiabilidade no processo, acabando por beneficiar ambas as partes. Além disso, tais mecanismos fomentam o crescimento econômico sustentável, visto que ampliam a divulgação e valorização dos profissionais e estabelecem uma uma padronização nas relações de prestação de serviço.  

Considerando essa demanda, este projeto propõe o desenvolvimento de uma plataforma web baseada na interação eficiente, na transparência e em critérios objetivos de qualidade e confiabilidade, voltada à divulgação e contratação de trabalhadores que prestam serviços sob demanda.  O modelo proposto vai além da simples intermediação de contratos, integrando mecanismos de verificação de competências e avaliações de qualidade dos serviços, o que aumenta a confiança entre as partes envolvidas. Assim, a proposta oferece uma solução que se adapta às exigências de um mercado de trabalho dinâmico e em constante evolução.